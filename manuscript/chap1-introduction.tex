\section{Introduction}   \label{chap:introduction}

> Goals:
Introduce the history of RTS, real-time control.
Introduce CPS
Introduce SWaP
Introduce growing need for research that combines RTS and Control.

Summary Statement: Real-time systems are systems which require logical and temporal correctness. Informally, the result of a requested computation must be correct (logical correctness) and it must be completed before the associated deadline.

\subsection{The Rise of Real-Time Systems}

Summary: Real-time systems are the follow up to real-time simulations. Post WWII simulations gave rise to real-time simulation. Managing telecommunications required real-time system modeling - the need for timely switching and routing of telephone signals to requested destination numbers. Eventually real-time systems were applied to industrial process control - the earliest of which was isomerization of butane in the chemical sector. In short, real-time systems began as a natural consequence of three things: simulating real-world control systems for military aircraft, serving user requests for telecommunications, and executing control of chemical process control.

\subsection{Modern Real-Time Systems}

Summary: Modern RTS are now found in the same areas they began: aircraft, simulators, telecommunications, and industrial processes and have extended into more spaces.
Real-time systems are found in autonomous (and non-autonomous vehicles), passenger aircraft, spacecraft, medical devices, and critical infrastructure.
Modern RTS are not only concerned with safely executing control systems but also with reducing the Size, Weight and Power (SWaP) of devices.
Modern RTS are also becoming more integrated with real-world dynamics by introducing .
With the advent of CPS, RTS and Control Systems are further integrated.
Modern RTS are not only converned with scheduling but also with SWaP.

\subsection{Research Need}
Growing number of devices, desire to lower SWaP, desire to integrate closely with physical world (CPS) require less pessimistic, more efficient, and more inclusive analysis.

\subsection{Approach}
Incorporate physical dynamics of the systems whose workloads are being scheduled so that physical dynamics may be leveraged for more efficient demand characterization and schedulability analysis.
Furthermore, incorporating physical dynamics will allow system designers to visualize tradeoffs between real-time metrics and physical system dynamics.

\subsection{Thesis}

> Pull from NSF/NDSEG

\subsection{Contributions}

The main contributions of this work are:
\begin{enumerate}
    \item a hardware-software codesign approach for software-based short-circuit detection that demonstrates the tradeoff of processor utilization under EDF against inductor size (and thus board space consumed by circuitry) and
    \item a demand characterization method for Adaptive Variable-Rate (AVR) tasks used in Internal Combustion Engines (ICEs) in which engine dynamics are used to limit the search space for the Demand Bound Function (DBF).
\end{enumerate}

\subsection{Scope of Contributions}

Two methods of sw-based short circuit detection with a codesign relationship between PCB board space consumed and real-time utilization required for a sw-based short circuit detection.
Knapsack-based approach to characterizing the demand of an engine control task where engine dynamics are 

\subsection{Organization}

The remainder of this work is as follows:

\begin{enumerate}
    \item Chapter \ref{chap:relatedWork} summarizes the related work for each major contribution.
    \item Chapter \ref{chap:systemModel} provides the system model and terms common to each contribution.
    \item Chapter \ref{chap:codesign} covers the codesign of software-based short circuit protection systems.
    \item Chapter \ref{chap:demandCharacterization} describes the demand characterization of AVR tasks in ICEs.
    \item Chapter \ref{chap:futureWork} discusses future work.
    \item Chapter \ref{chap:conclusion} summarizes and concludes this work.
    \item Chapter \ref{chap:publicationList} lists the publications which contributed to this work.
\end{enumerate}