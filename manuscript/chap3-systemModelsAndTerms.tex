\section{System Models and Terms}   \label{chap:systemModel}

This chapter will describe common terminology for describing computation, timing, and analysis in the context of real-time systems.
Note that any terminology or notation specific to individual systems (for example, engine control) are not covered here but will be covered in their respective chapters.

\subsection{Modeling Computation: Jobs and Tasks}

To properly characterize demand or perform schedulability analysis, the computational load of a real-time system must be described.
The combination of \textit{jobs} and \textit{tasks} allow us to do so.

\subsubsection{Jobs}

A job, $j_i$, is the smallest unit for modeling real-time computation characterized by the tuple $j = (a,e,d)$ where $a$ is the release time, the earliest time at which the processor time may be allocated to the job, $d$ is the relative deadline, the time by which $e$ units of processor time must be allocated to the job to avoid missing the deadline.
Figure \ref{fig:rt-job} illustrates a single job and its typical parameters.

\subsubsection{Tasks}

A task, $\tau_i$, is an infinite series of jobs characterized by the tuple $\tau_i = (a,p,c,d)$ where $a$ is the offset of the task, the time after $t=0$ at which the first job of the task is released, $p$ is either the period or minimum interarrival time of the task (described below), $c$ is the WCET of all jobs the task may release, and $d$ is the relative deadline for all jobs of the task such that a job released at time $t$ is due at time $t+d$.

Tasks may be either aperiodic (also known as sporadic) or periodic.
Sporadic tasks release jobs at irregular intervals in which case $p$ represents the minimum interarrival time between successive jobs.
Periodic tasks release jobs at regular intervals in which case $p$ represents the fixed interarrival time between successive jobs.
Figure \ref{fig:rt-task} illustrates two tasks, one sporadic and one periodic, and their typical parameters.

\subsubsection{Task Set}

When more than one task is needed to describe all computational loads, tasks are represented by a task set, $\Tau = \{\tau_1, \tau_2, \dots, \tau_n\}$, a collection of individual tasks.
From this task set, an additional parameter, the Hyperperiod $H$, can be derived.
The hyperperiod $H$ represents the least common multiple (LCM) of all periods (or minimum interarrival times) in  the task set.
Formally, $H = \text{LCM}(p_1, p_2, \dots, p_n)$.
This hyperperiod represents when pattern of computation repeats.

\subsection{Bounding Computation: Utilization and Demand}

With the fundamental tools for modeling computation described, we now describe tools to represent bounds on the total computation a task (or set of tasks) may require. 

\subsubsection{Utilization}

One method of bounding load is utilization, the ratio of WCET to period.
The utilization for an individual task is thus $u_i = \frac{c_i}{p_i}$.
The utilization for a task set is then,
\begin{equation}
    U = \sum_{i \in \Tau} \frac{c_i}{p_i}.
\end{equation}

Note that while utilization describes the ratio of time consumed to time available, it does not describe the change in computational load over time as jobs are released or job deadlines pass.

\subsubsection{Demand and the Demand Bound Function}

To provide a more precise bound on computation over time, demand is used.
The demand over some time interval $[t_1,t_2]$ is the sum of all WCETs of jobs with release time and deadline in the interval.
Demand, however, only reflects a particular interval and not any possible interval.
The Demand Bound Function (DBF), introduced by Baruah et al. \cite{baruah_preemptively_1990}, is a function which characterizes demand by providing the maximum cumulative execution time a set of tasks may require from a processor over any interval of size $\delta$.
The definition presented in Baruah et al. \cite{baruah_preemptively_1990} is used here.
\begin{definition}[Demand bound Function]\label{def:dbf}
    The \textit{demand bound function}, $DBF(\tau,\delta)$, gives the cumulative WCET of all jobs of $\tau$ with both release times and deadlines within any time interval of length $\delta$.
\end{definition}

\subsection{Schedules, Feasibility, Scheduling Algorithms and Schedulability}

With utilization and the DBF to bound computation, we now introduce the schedule.

\subsubsection{Schedule}

A function describing which jobs or tasks are executed.
A map of workloads onto available time.

\subsubsection{Utilization}

\subsubsection{Demand Bound Function}


\subsubsection{Scheduling Algorithms}

An algorithm which takes tasks as inputs and generates a schedule.


