\section{Related Work}   \label{chap:relatedWork}

In this chapter, we introduce works related to the software-based short circuit detection methods and the demand characterization of engine control tasks.
In the context of short circuit detection, many works exist proposing variants of hardware-based detection methods but none present a real-time software-based approach.
In the context of engine control tasks, several works have covered the worst-case demand of Adaptive Variable Rate (AVR) tasks but none have provided a computation as efficient as ours.

\subsection{Short Circuit Detection} \label{subsec:scd-relatedWork}

To the best of our knowledge, current approaches to short-circuit protection are rooted in dedicated circuitry and do not use adaptive real-time processing.
Modern methodologies include Zone Selective Interlocking on systems with alternating current (AC) which, while successful, cannot be directly applied to DC \cite{du_new_2014}.
Other methods use the Rogowski coil in conjunction with differential and integral signals from the coil to detect a short \cite{wang_new_2011}.
While feasible, the Rogowski coil implementation is large and does not slow current rise in the system.
\textbf{TODO: Double check that "reference" here isn't the PIC ref.} Some methods only require sampling of gate emitter voltage and a reference voltage \cite{horiguchi_short_2014}.
Krone et al. use the gate emitter voltage, but also reference the DC link capacitors to assist in protection \cite{krone_fast_2015}.
Each method, while different, relies solely on dedicated circuitry for short protection.
Although variations exist in the components used and the design of the circuits, we sought an alternative where required circuitry was minimized.

Furthermore, protection methods relying on the change in current, as in Hain and Bakran \cite{hain_new_2016}, require an inductor.
This approach also includes auxiliary MOSFETs, latch circuits, and comparators.
Additionally, some current protection methods relying on gate charge require differential amplifiers attached to auxiliary MOSFETs as found in Horiguchi et al. which also require an inductor in test circuits \cite{horiguchi_high-speed_2015}.
Both models have more components than the single inductor required for protection in this approach, providing more motivation for software-based protection.

In the area of cyber-physical and real-time systems, the senior thesis \cite{willcock_short_2016} upon which this work and its conference-published variant \cite{willcock_trading_2017-1} are based focused primarily on establishing a relationship between the inductance of an inductor and the utilization required for the task.
This work extends the results of Willcock \cite{willcock_short_2016} by incorporating board space consumed into the utilization calculation and providing experimental validation of the extended relationship with low-cost hardware. Excluding the preceding variant of this work, we are unaware of other cyber-physical or real-time works specific to short protection. However, works identifying adaptive real-time tasks with multiple operating modes are present \cite{huang_techniques_2015}. Examples include thermal-aware computing in Hettiarachchi et al \cite{hettiarachchi_design_2014} and rate-adaptive tasks as in Buttazzo et al. \cite{buttazzo_rate-adaptive_2014}. Biondi and Buttazzo furthered this model with thorough analysis of its implications on the executing processor \cite{biondi_engine_2015}.
These works address properties of environment and power-aware real-time tasks but are not specific to short-circuit protection.

\subsection{Engine Control} \label{subsec:engCtrl-relatedWork}

Usage of multiple worst-case execution times and periods for engine-controlled tasks was first studied by Kim et al. \cite{kim_rhythmic_2012}, where the authors proposed the rhythmic task model and obtained schedulability results assuming the dependency of a task's attributes on external physical events.
However, the analysis is limited to a single AVR task scheduled along with periodic tasks using rate monotonic scheduling algorithm in which the AVR task has the highest priority.

Biondi et al. \cite{biondi_exact_2014} presented the calculation of the worst-case demand as a search problem in the speed domain and the infinite number of paths in the search tree was narrowed down by identifying certain paths that met a given criteria.
A similar method was applied using rate monotonic~\cite{biondi_response-time_2015} and EDF scheduling~\cite{biondi_feasibility_2015}.
However, these works assume a constant acceleration between two jobs releases, which does not always result in the worst-case demand, as shown by Mohaqeqi et al. \cite{mohaqeqi_refinement_2017}.
In one of the first works on EDF scheduling of AVR tasks, Guo and Baruah~\cite{guo_uniprocessor_2015} developed a sufficient schedulability test based on a speed-up factor analysis. %\sandeep{Remove the description of [9] and [11] if needed for space.}

Identifying the fact that the exact speed of rotation of the crankshaft may not be known, Biondi et al. \cite{biondi_real-time_2016} proposed two methods to estimate the angular speed of the crankshaft.
In a recent paper~\cite{biondi_real-time_2016}, Biondi et al. proposed a task model for expressing some practical features of engine control tasks and presented schedulability tests for engine control applications under EDF scheduling. 

A complementary direction of research on AVR tasks was undertaken by Biondi et al.~\cite{biondi_performance-driven_2016}, and focused on finding the boundary speeds of the modes to maximize the performance of the engine using an optimization based approach.

Mohaqeqi et al. \cite{mohaqeqi_refinement_2017} partitioned the speed domain and constructed the corresponding digraph real-time (DRT) task graph to determine the worst-case demand of the AVR task by searching from each of the nodes of the DRT graph. Though such an approach gives the exact value of the worst-case demand of the AVR task, it considers many unnecessary paths, resulting in long computation times. 
In this paper, we propose an algorithm to obtain the speed partitioning and to select a smaller subset of the paths considered by Mohaqeqi et al.~\cite{mohaqeqi_refinement_2017} to significantly reduce the computation time.

%Though knapsack-based algorithms are used in other areas of research~\cite{cloudKnapsack,loadKnapsack,mobileKnapsack}, to the best knowledge of the authors, this is the first instance of its application to calculate the maximum demand of AVR tasks.