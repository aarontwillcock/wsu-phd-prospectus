%Use unnumbered section for abstract
\section*{ABSTRACT}

%Add reference to the table of contents {toc} at the section level {section} titled "Abstract" {Abstract}
\addcontentsline{toc}{section}{Abstract}
\centerline{\bf Demand Characterization and Codesign of }
\vspace{-0.4cm}
\centerline{\bf Dynamics-Adaptive Real-Time Systems}

{\setlength\baselineskip{0.3in}
\begin{center}
by\\
\medskip
{\bf AARON WILLCOCK}\\
\medskip
{\bf MAY 2021}\\
\end{center}
\Vspc
\begin{tabular}{ll}
	{\bf Advisor:} & DR. FISHER \\
	{\bf Major:} & COMPUTER SCIENCE \\
	{\bf Degree:} & DOCTOR OF PHILOSOPHY
\end{tabular}
}

\bigskip \bigskip

The relationships between real-time parameters and physical system dynamics are studied and exploited to reduce the complexity of demand characterization and increase schedulability.
Example systems evaluated include software-based short-circuit detection and spark-ignition internal combustion engines (ICEs).
Exploited system dynamics are used to connect real-time utilization to printed circuit board space consumed by an inductor for short-circuit detection and reduce the search space when constructing Demand Bound Functions (DBFs) for ICEs.
Future works will exploit relationships between incoming and outgoing current in monitored power distribution systems, approximate the existing knapsack approach for constructing DBFs for ICEs, exploit variable displacement techniques in ICEs such as dynamic-skip fire, and consider demand-based reconfiguration of batteries.
The goal of this work and future works is to illustrate how knowledge about physical dynamics can be applied to real-time analysis to improve schedulability, reduce the complexity of assessing schedulability, and improve codesign of similar systems.